\ifdefined\texrc@package\else
\def\texrc@package{LOADED}

% iftex package
%% \ifXeTeX, \ifPDFTeX, \ifLuaTeX for selecting proper code for different engineer
%% \RequireXeTeX, \RequirePDFTeX, \RequireLuaTeX for specific the engineer for some codes
\usepackage{iftex}

% <tipa> need load before <ctex> because <ctex> package will load <fontspec> package that 
% loade before <tipa> package will cause error in xetex and luatex, 
\usepackage{tipa} % this is still erroneous in XeTeX and LuaTeX

% chinese environment. remember writing code with utf8, which can display
% chinese correctly, otherwise it will wrong.
\usepackage[UTF8]{ctex}

% hyperref package, which provide hyperref in documents.
% It provide command \pdfbookmark, and the usage of this command is:
% <\pdfbookmark[level]{bookmark text}{anchor name}>
\usepackage{hyperref}

% keyval package
% This package implements a system of defining and using sets of parameters,
% which are set using the syntax <key>=<value>.
% For each keyword in such a set, there exists a function which is called whenever
% the parameter appears in a parameter list. For instance if the set dpc is to have
% the keyword scale then I would define.
\usepackage{keyval}

% bookmark package, which is used to create bookmark similar to 
% the function provided by {hyperref} package.
% usage is:
% \bookmarksetup{<options>}
% \bookmarksetupnext{<options>}, which only affect next \bookmark command
% \bookmark{<options>}{title}
\usepackage{bookmark}

% pdf page
\usepackage{pdfpages}

% provide color support
\usepackage{xcolor}

% index
\usepackage{imakeidx}

% geometry package
% This package provides a flexible and easy interface to page dimensions.
% You can change the page layout with intuitive parameters. 
% For instance, if you want to set a margin to 2cm from each edge of
% the paper, you can type just \usepackage[margin=2cm]{geometry}. 
% The page layout can be changed in the middle of the document 
% with \newgeometry command.
\usepackage{geometry}

% margin note package
\usepackage{marginnote}

% listings macro for code
\ifdefined\no@need@listings\else
\usepackage{listings}
\fi
\usepackage[sharp]{easylist}

% graphicx
\usepackage{graphicx}

% comment, provide \begin{comment}...\end{comment} environment
\usepackage{comment}

% asymptote support
\iffalse
\usepackage{asypictureB}
\fi

% specify the title format, commands:
% 1, \titlelabel{...}, \thetitle will represent such 1.2, 1.3 and so on.
% 2, \titleformat*{command}{format} will change format of next title,
% here <command> will be <\part>, <\chapter>, <\section>, <\subsection>,
% <\subsubsection>, <\paragragh>, <\subparagragh>.
% 3, \titleformat{command}[shape]{format}{label}{sep}{befor-code}[after-code], 
% here <command> is same with above. 
% 4, \titleclass{name}{class}[super-level-class], \titleclass{\subchapter}{straight}[\chapter]
% also, need define new macro to complement that. \newcounter{subchapter} and 
% \renewcommand{\thesubchapter}{\Alph{subchapter}
% 5, \titlerule[<height>], \titleline[<align>lcr]{horizonal material}, \tilterule*[<height>]{text}
\usepackage{titlesec}

%\usepacakge{fontspec}

\usepackage{algorithm}
\usepackage{algorithmicx}
\usepackage{algpseudocode}

%{{{ tabular class
\usepackage{array}

\usepackage{longtable}

% wrap line in columns
\usepackage[english]{babel}
%}}}

%{{{ Float
% float
\usepackage{float}

% side capture
\usepackage{sidecap}

% wrap figure
\usepackage{wrapfig}
% wrapfigure environment

% sub caption, it include <caption> package
\usepackage{subcaption}
%}}}

%{{{ math
% amsthm package for theorem
\usepackage{amsthm}

% amsmath package
\usepackage{amsmath}

% mathrsfs package
\usepackage{mathrsfs}

% amsfonts
\usepackage{amsfonts}

% tensor
\usepackage{tensor}

\usepackage{fixmath}
\usepackage{bm}
%}}} end math

% directed drawing graph
\usepackage{tikz}

\usepackage{filecontents}

\usepackage{blindtext} %% to add dummy text

\usepackage{xstring}

\usepackage{calc}
\fi
